%\section{Annexe}
%Ce troisième chapitre s'intéresse à présenter le domaine de workflow scientifique.

%\subsection{workflow scientifique}
%Le concept de workflow structure le processus de simulation scientifique comme un graph d'activities, dans lequel les noeuds correspondent aux activitiés de calcul numérique et les arêtes représentent le flux de données entre eux. les activités du workflow sont associées aux programmes qui préparent et analysent les données\cite{oga11}, on appel ces programmes des activations. 

% \vspace{0.25cm}
%\subsection{Les systèmes de gestion de workflow scientifique}
%Ce sont des systèmes de gestion de flux de travail (scientifique workflow management system) qui supportent la définition, l'exécution et la simulation des workflow scientifique\cite{oga11}.

\appendix
\chapter{Logiciels}
Durant ce stage on s'appuie principalement sur deux logiciels :
\begin{description}\label{chiron}
\item[Chiron]  \url{http://chironengine.sourceforge.net/}.

Chiron est un système de gestion de flux de travail de calculs numériques (\textit{scientifique workflow management system}),  il exécute ces simulations comme une chaîne d'activités (programmes) et un flux de données (dataflow) sur ces activités. Ce système fournit la gestion des simulations scientifiques, leur exécution parallèle tout en enregistrant la provenance des données. Chiron implémente l'approche algébrique dans un style MapReduce. L'utilisation de MapReduce comme approche de programmation permet aux scientifiques de programmer d'une façon plus simple la procédure du calcul en cachant le parallélisme, qui peut être complexe à gérer \cite{oga13}.


\item[SPOC] \url{http://www.algo-prog.info/spoc/}.

SPOC (Stream Processing with OCaml)  consiste en une extension à OCaml associée à une bibliothèque d'exécution. L'extension permet la déclaration de noyaux GPGPU externes utilisables depuis un programme OCaml, tandis que la bibliothèque permet de manipuler ces noyaux ainsi que l'automatisation des transferts de données nécessaires à leur exécution. SPOC offre, de plus, une abstraction supplémentaire en unifiant les deux environnements de développement GPGPU (Cuda et OpenCL) en une même bibliothèque \cite{bou14}, \cite{bour14}.


\end{description}
\vspace{0.5cm}

%\begin{description} c'est quoi la logiciel
%\item[GPU] NVIDIA Corporation GF108GLM [Quadro 1000M].
%\end{description}
\vspace{\parskip}
%\end{flushleft}
