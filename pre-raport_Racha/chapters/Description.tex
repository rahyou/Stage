\chapter{Introduction} % Pas de numérotation
%\addcontentsline{toc}{section}{Description} % Ajout dans la table des matières
%\paragraph{Introduction}

Ce pré-rapport décrit mon stage actuellement effectué au sein de l'équipe  APR (Algorithmes, Programmes et Résolution) au laboratoire LIP6 à l'Université Pierre-Et-Marie-Curie sous la direction de M. Emmanuel Chailloux et M. Mathias Bourgoin. Le stage de fin d'étude s'inscrit dans le cadre du Master d'informatique spécialité Systèmes et Applications Répartis (SAR). Il se déroule du 2 février au 31 Juillet 2016.

L'intitulé de mon stage est  \og Exécution parallèle sur cartes graphiques d'un système de gestion de flux de travail de calculs numériques\fg{}, Ce projet est en collaboration avec les universités brésiliennes UFRJ\footnote{UFRJ -   l'Université Fédérale de Rio de Janeiro} et UFF\footnote{UFF - l'Université fédérale Fluminense.}.
\noindent
%\subsection{Organisation du document}
%Ce pré-rapport présente d'abord les systèmes de gestion de workflow scientifique, l'approche algébrique adoptée par ces systèmes, et la programmation GPGPU. Il décrit le contexte de la mission et définit clairement les objectifs de mon stage et sa roadmap générale.

%Une annexe contient les logiciels utilisés dans ce stage.



