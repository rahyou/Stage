
\section{ Planification du travail}
Le stage est organisé selon le plan général suivante:
\subsection{Planification du travail}
\begin{itemize}
\item [\textbullet] Février 2016:
	\begin{enumerate}
	\item État de l'art.
	\item Sélectionner et installer la plateforme de travail.
	\end{enumerate}
\item [\textbullet] Mars 2016 :
	\begin{enumerate}
	\item État de l'art.
	\item Analyser et étudier Chiron. 
	\item S'initier à la programmation fonctionnelle, impérative, et orientée objet avec Ocaml.
	\end{enumerate}
\item [\textbullet] Avril 2016 : 
	\begin{enumerate}
	\item Exécuter des flux de travails de Chiron sur la carte graphique.
	\item Exécuter des flux de travails de chiron sur multi-gpu. 
	\end{enumerate}
\item [\textbullet] Mai 2016 :
\begin{enumerate}
	\item Réflexion sur l'algorithme de l'optimisation et de la planification d'exécution parallèle.
	\item Mise en {\oe}uvre d'une version fonctionnelle de Chiron.
	\end{enumerate}
\item [\textbullet] Juin 2016 :
\begin{enumerate}
\item Phase d'expérimentations.
	\item Comparaison entre les deux approches (Chiron de base, Chiron amélioré).
	\end{enumerate}
\item [\textbullet] Juillet 2016 :
\begin{enumerate}
	\item Rédaction. 
	\end{enumerate}	
\end{itemize}

\subsection{Prospective}

Étudier la possibilité de parallélisation des opérations de Chiron, pour introduire une nouvelle version de Chiron qui consiste à combiner les fonctionnalités de Chiron et celles de SPOC.
